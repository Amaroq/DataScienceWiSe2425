% DOCUMENT CLASS %
\documentclass{article}

% PACKAGES%
\usepackage{amsfonts}   % For a basic mathfont (many will be replaced by stix)
\usepackage{amsmath}    % For basic math symbols
\usepackage{bbm}        % For \mathbbm (better version of \mathbb)
\usepackage{mathrsfs}   % For \mathscr
\usepackage{hyperref}   % For footnotes
\usepackage{csquotes}   % For \textquote{}
\usepackage{IEEEtrantools}  % For better alignments
\usepackage{tikz}   % For a macro
\usepackage{listings} % For code
\usepackage{xcolor}   % For ... code
\usepackage[utf8]{inputenc}  % Erlaubt es, Umlaute etc. zu verwenden (Datei muss UTF-8 Kodierung haben)
\usepackage[ngermanb]{babel} % Deutsche Übersetzung und Silbentrennung (Neue Rechtschreibung)

\usepackage{enumitem}
\setlist[itemize]{noitemsep}

\usepackage{fontspec}
\usepackage{unicode-math}

% FONT CONFIGURATION %
\setmainfont{Crimson Text}[
  BoldFont={Crimson Text Bold},
  SmallCapsFont={Times Small Caps & Old Style Fi},
  Ligatures={TeX,Common}
]
\setmathfont{Latin Modern Math}[Scale=MatchUppercase, FakeBold={2}]

% SETS %
\newcommand*{\R}{\mathbbm R}    % Set of real numbers
\newcommand*{\N}{\mathbbm N}    % Set of natural numbers (beginning at 0)
\newcommand*{\Z}{\mathbbm Z}    % Set of integers
\newcommand*{\Q}{\mathbbm Q}    % Set of rational numbers
\newcommand*{\pro}{\mathcal P}  % Power set of a set
\newcommand*{\Co}{\mathcal C^1}     % Set of R^n -> R^d functions that are fully partially differenciable and continuous
\newcommand*{\Ct}{\mathcal C^2}     % Set of R^n -> R functions that are C^1 functions and whose gradient is C^1 function
\newcommand*{\SetP}{\textup{P}}
\newcommand*{\SetNP}{\textup{NP}}
\newcommand*{\SetSAT}{\textup{SAT}}
\newcommand*{\SetClique}{\textup{Clique}}
\newcommand*{\SetKClique}{k\smi\textup{Clique}}
\newcommand*{\SetKColor}{k\smi\textup{Color}}


% OPERATIONS ON FUNCTIONS %
\newcommand*{\ddx}{\frac{\text d}{\text dx}}    % Derivative of a function
\newcommand*{\grad}{\text{grad}}    % Gradient of a function
\newcommand*{\J}{\textbf{J}}        % Jacobi-matrix of a function
\newcommand*{\He}{\textbf{H}}       % Hesse-matrix of a function
\newcommand*{\LMAX}{\text{\scshape Lmax}}   % Set of maxima of a function
\newcommand*{\LMIN}{\text{\scshape Lmin}}   % Set of minima of a function
\newcommand*{\img}{\text{img}}      % Image of a function
\newcommand*{\dom}{\text{dom}}      % Domain of a function

% DISTRIBUTIONS %
\newcommand*{\Unif}{\text{Unif}}    % Uniform distribution (discrete or continous)
\newcommand*{\Ber}{\text{Ber}}  % Bernoulli distribution
\newcommand*{\Bin}{\text{Bin}}  % Binomial distribution
\newcommand*{\Exp}{\text{Exp}}  % Exponential distribution
\newcommand*{\Geo}{\text{Geo}}  % Geometric distribution
\newcommand*{\Pois}{\text{Pois}}    % Poisson distribution
\newcommand*{\Norm}{\mathcal{N}}    % Normal distribution
\newcommand*{\D}{\mathcal{D}}   % Some random distribution

% BASIC PROBABILISTIC STUFF %
\newcommand*{\E}{\mathcal E}    % Some space of events
\newcommand*{\Pro}{\mathbbm P}  % Probability function of an event
\newcommand*{\Od}{\text{Od}}    % Odds of an event

% OPERATIONS ON DISTRIBUTIONS OR RANDOM VALUES %
\newcommand*{\supp}{\text{supp}}    % Support of a random variable
\newcommand*{\Ew}{\mathbbm E}   % Expected value
\newcommand*{\Var}{\text{Var}}  % Variance
\newcommand*{\Cov}{\text{Cov}}  % Covariance of two random variables
\newcommand*{\B}{\mathbbm B}    % Bias of a point estimation

% MACROS %
\newcommand{\subsubsubsection}[2]{\textsc{\underbar{#1}} \\#2\\\\}  % Macro for title in ALLCAPS, ends with two newlines
\newcommand*{\todo}{\textit{\dots TODO \dots}}  % Macro for ... todo ...
\newcommand*{\QDp}{{:}\text{ }} % Macro for ": " so that when writing something like "\forall x:" there is not a seperation between the '\forall' and the ':'
\newcommand*{\puffer}{\text{ }\text{ }\text{ }\text{ }} % Macro for a lazy aligment 1
\newcommand*{\gedanke}{\textbf{-- }}    % Macro for a long minus followed by text
\newcommand*{\smi}{\text{-}}        % Macro for a small minus ('-')
\newcommand*{\gap}{\text{ }}    % Macro for a lazy aligment 2
\newcommand*{\qed}{\null\nobreak\hfill\ensuremath{\square}} % Macro for a QED-Box
% for a funny ¯\_(ツ)_/¯-Emoji
\newcommand{\shrug}[1][]{%
    \begin{tikzpicture}[baseline,x=0.8\ht\strutbox,y=0.8\ht\strutbox,line width=0.125ex,#1]
    \def\arm{(-2.5,0.95) to (-2,0.95) (-1.9,1) to (-1.5,0) (-1.35,0) to (-0.8,0)};
    \draw \arm;
    \draw[xscale=-1] \arm;
    \def\headpart{(0.6,0) arc[start angle=-40, end angle=40,x radius=0.6,y radius=0.8]};
    \draw \headpart;
    \draw[xscale=-1] \headpart;
    \def\eye{(-0.075,0.15) .. controls (0.02,0) .. (0.075,-0.15)};
    \draw[shift={(-0.3,0.8)}] \eye;
    \draw[shift={(0,0.85)}] \eye;
    % draw mouth
    \draw (-0.1,0.2) to [out=15,in=-100] (0.4,0.95); 
    \end{tikzpicture}
}

% ADJUSTMENTS OF PAPER %                                                                                                                                                                                                                
\setlength{\hoffset}{-1.5cm}
\setlength{\voffset}{-2.5cm}
\setlength{\textheight}{24cm} % DIN A4 ~ 30cm
\setlength{\textwidth}{15cm}  % DIN A4 = 21cm, 18 suffice.

% Für deutsche Dokumente
\renewcommand{\contentsname}{Inhalt}
\renewcommand{\partname}{Teil}

\lstdefinestyle{mystyle}{
    basicstyle=\ttfamily\footnotesize,  % the size of the fonts that are used for the code
    breakatwhitespace=false,            % sets if automatic breaks should only happen at whitespace
    breaklines=true,                    % sets automatic line breaking
    frame=leftline,
    keepspaces=true,                    % keeps spaces in text, useful for keeping indentation of code (possibly needs columns=flexible)
    numbers=left,                       % where to put the line-numbers; possible values are (none, left, right)
    numbersep=5pt,                      % how far the line-numbers are from the code
    showspaces=false,                   % show spaces everywhere adding particular underscores; it overrides 'showstringspaces'
    showstringspaces=false,             % underline spaces within strings only
    showtabs=false,                     % show tabs within strings adding particular underscores
    tabsize=2,                          % sets default tabsize to 4 spaces
    commentstyle=\sffamily\itshape,
    emph={*, do, od, for, if, fi, then, else, to, def, in, forall, exists, while}, 
    emphstyle={\sffamily\bfseries},
    keywordstyle={\sffamily\bfseries},   % emphasis style for the keywords,
    texcl=true
}

\lstset{style=mystyle}



\newcommand{\nr}{4}
\title{Data Science Assignment \nr}
\author{Nike Marie Pulow -- Henri Paul Heyden \\ \small stu239549 -- stu240825}
\date{}

\begin{document}
    \maketitle
    \section{PCA, SVD}
    \begin{itemize}
        \item[1.]
            We search for matrices \(U, \Sigma, V\) with \(X = U \Sigma V^T\) so that the columns of \(U\) and \(V\) are eigenvectors of \(XX^T\) and \(X^TX\) respectively. \\
            We'll begin by computing \(XX^T\): \\
            \begin{IEEEeqnarray*}{r'c'l}
                XX^T & = & \begin{bmatrix}
                                4 & 2 \\
                                \sqrt 2 & 2 \sqrt 2 \\
                                -\sqrt 2 & -2 \sqrt 2
                           \end{bmatrix} \times
                           \begin{bmatrix}
                                4 & \sqrt 2 & -\sqrt 2 \\
                                2 & 2 \sqrt 2 & -2 \sqrt 2
                           \end{bmatrix} \\
                    & = & \begin{bmatrix}
                                20 & 8 \sqrt 2 & -8 \sqrt 2 \\
                                8 \sqrt 2 & 10 & -10 \\
                                -8 \sqrt 2 & -10 & 10
                          \end{bmatrix}
            \end{IEEEeqnarray*}
            Now we have to solve \(\det(X - \lambda I_{3 \times 3}) = 0\) for the eigenvalues of \(XX^T\). \\
            An expanded form of this equation is:
            \begin{IEEEeqnarray*}{r'c'l}
                0 & = & \det(X - \lambda I_{3 \times 3}) \\
                  & = & \det\left(
                            \begin{bmatrix}
                                20 & 8 \sqrt 2 & -8 \sqrt 2 \\
                                8 \sqrt 2 & 10 & -10 \\
                                -8 \sqrt 2 & -10 & 10
                            \end{bmatrix} -
                            \begin{bmatrix}
                                \lambda & 0 & 0 \\
                                0 & \lambda & 0 \\
                                0 & 0 & \lambda
                            \end{bmatrix}
                            \right) \\
                  & = & \det\left(
                            \begin{bmatrix}
                                20 - \lambda & 8 \sqrt 2 & -8 \sqrt 2 \\
                                8 \sqrt 2 & 10 - \lambda & -10 \\
                                -8 \sqrt 2 & -10 & 10 - \lambda
                            \end{bmatrix}
                            \right) \\
                  & = & (20 - \lambda)(10 - \lambda)(10 - \lambda) + (8 \sqrt 2)(-10)(-8 \sqrt 2) + (-8 \sqrt 2)(8 \sqrt 2)(-10) \\
                    & & - (-8 \sqrt 2)(10 - \lambda)(-8 \sqrt 2) - (8 \sqrt 2)(8 \sqrt 2)(10 - \lambda) - (20 - \lambda)(-10)(-10)
            \end{IEEEeqnarray*}
            This is a third degree polynomial equation with expanded form \(-x^3 + 40x^2 - 144x = 0\) \footnote{Actually writing each step of this down would take too much space.} \\
            We get the solutions: \[\lambda_1 = 36 \wedge \lambda_2 = 4 \wedge \lambda_3 = 0\]
            With this we have \[\Sigma = \begin{bmatrix}
                                            6 & 0 & 0 \\
                                            0 & 2 & 0 \\
                                            0 & 0 & 0
                                          \end{bmatrix}\]
            With the eigenvalues of \(XX^T\) being calculated, we can now solve for the eigenvectors of \(XX^T\):
            \begin{IEEEeqnarray*}{r'c'l}
                \begin{bmatrix}
                    0 \\
                    0 \\
                    0
                \end{bmatrix}
                & = &
                \left(
                \begin{bmatrix}
                    20 & 8 \sqrt 2 & -8 \sqrt 2 \\
                    8 \sqrt 2 & 10 & -10 \\
                    -8 \sqrt 2 & -10 & 10
                \end{bmatrix}
                -
                \begin{bmatrix}
                36 & 0 & 0 \\
                0 & 36 & 0 \\
                0 & 0 & 36
                \end{bmatrix}
                \right)
                \times
                \begin{bmatrix}
                    {v_1}_1 \\
                    {v_1}_2 \\
                    {v_1}_3
                \end{bmatrix} \\
                & = &
                \begin{bmatrix}
                    -16 & 8 \sqrt 2 & -8 \sqrt 2 \\
                    8 \sqrt 2 & -26 & -10 \\
                    -8 \sqrt 2 & -10 & -26
                \end{bmatrix}
                \times
                \begin{bmatrix}
                    {v_1}_1 \\
                    {v_1}_2 \\
                    {v_1}_3
                \end{bmatrix} \\
                & = &
                \begin{bmatrix}
                    -16 {v_1}_1 + 8 \sqrt 2 {v_1}_2 - 8 \sqrt 2 {v_1}_3 \\
                    8 \sqrt 2 {v_1}_1 - 26 {v_1}_2 - 10{v_1}_3 \\
                    -8 \sqrt 2 {v_1}_1 - 10 {v_1}_2 - 26 {v_1}_3
                \end{bmatrix}
            \end{IEEEeqnarray*}
            When numbering the equations in collumns \textbf{I}, \textbf{II} and \textbf{III} we can see that \(\textbf{I} \Rightarrow {v_1}_1 = \frac{\sqrt 2 {v_1}_2}{2} - \frac{\sqrt 2 {v_1}_3}{2}\) and \\
            \(\textbf{II} + \textbf{III} \Rightarrow {v_1}_2 = -{v_1}_3\) which results in:
            \[{v_1}_1 = \sqrt 2 {v_1}_2\] and respectively:
            \[{v_1}_2 = \frac{{v_1}_1}{\sqrt 2} \wedge {v_1}_3 = -\frac{{v_1}_1}{\sqrt 2}\]
            There are infinite solutions for this, with one being: \[{v_1}_1 = 1 \wedge {v_1}_2 = \sqrt 2^{-1} \wedge {v_1}_3 = -\sqrt 2^{-1}\]
            So we get one eigenvector:
            \[v_1 = \begin{bmatrix}
                1 \\
                \sqrt 2^{-1} \\
                -\sqrt 2^{-1}
            \end{bmatrix}\]
            Because writing the processes down takes a lot of space, the following linear equation systems will not be solved, but solutions provided. We have already demonstrated that we can solve one and the probability of minor mistakes when writing it down ruining the whole task is too great. \\
            The second eigenvector we get by this time subtracting the diagonal matrix times \(4\) instead of \(36\) and we get a solution:
            \[v_2 = \begin{bmatrix}
               1 \\
               -\sqrt 2^{-1} \\
               \sqrt 2^{-1}
            \end{bmatrix}\]
            And the third eigenvector is quite trivial:
            \[v_3 = \begin{bmatrix}
               0 \\
               1 \\
               1
            \end{bmatrix}\]
            Something that wasn't specified in the slides was that the eigenvectors must be normalized so that \(U\) and \(V\) are orthonormal. \\
            For that we have to divide all these vectors by \(\sqrt 2\). \\
            We get:
            \[U = 
                \begin{bmatrix}
                    \sqrt 2^{-1} & \sqrt 2^{-1} & 0 \\
                    1/2 & -1/2 & \sqrt 2^{-1} \\
                    -1/2 & 1/2 & \sqrt 2^{-1}
                \end{bmatrix}
            \]
            For a quick verification, we can compute \(UU^T\) and yes we get \(I_{3 \times 3}\), so \(U\) is indeed orthonormal. \\
            Now to \(V\) \\
            As the lecture specifies, the eigenvalues of \(XX^T\) are the same like those of \(X^TX\). \\
            This means we \textquote{just} have to solve the linear equations but for \(X^TX\) and not \(XX^T\). \\
            We'll begin by computing \(X^TX\): \\
            \begin{IEEEeqnarray*}{r'c'l}
                X^TX & = & \begin{bmatrix}
                                4 & \sqrt 2 & -\sqrt 2 \\
                                2 & 2 \sqrt 2 & -2 \sqrt 2
                           \end{bmatrix} \times
                           \begin{bmatrix}
                                4 & 2 \\
                                \sqrt 2 & 2 \sqrt 2 \\
                                -\sqrt 2 & -2 \sqrt 2
                           \end{bmatrix}\\
                    & = & \begin{bmatrix}
                                20 & 16 \\
                                16 & 20
                          \end{bmatrix}
            \end{IEEEeqnarray*}
            Thank god that this time it's small and regular. \\
            We get the following systems of linear equations:
            With eigenvalue \(36\) we get:
            \begin{IEEEeqnarray*}{r'c'l}
                \begin{bmatrix}
                    0 \\
                    0 \\
                    0
                \end{bmatrix}
                & = &
                \begin{bmatrix}
                    -16 & 16 \\
                    16 & -16
                \end{bmatrix} \times
                \begin{bmatrix}
                    {v_1}_1 \\
                    {v_1}_2
                \end{bmatrix} \\
                & = &
                \begin{bmatrix}
                    16 {v_1}_1 - 16 {v_1}_2 \\
                    16 {v_1}_1 - 16 {v_1}_2 \\
                \end{bmatrix}
            \end{IEEEeqnarray*}
            With one solution being: \[v_1 = \begin{bmatrix} \sqrt 2^{-1} \\ \sqrt 2^{-1} \end{bmatrix}\]
            With eigenvalue \(4\) we get:
            \begin{IEEEeqnarray*}{r'c'l}
                \begin{bmatrix}
                    0 \\
                    0 \\
                    0
                \end{bmatrix}
                & = &
                \begin{bmatrix}
                    16 & 16 \\
                    16 & 16
                \end{bmatrix} \times
                \begin{bmatrix}
                    {v_2}_1 \\
                    {v_2}_2
                \end{bmatrix} \\
                & = &
                \begin{bmatrix}
                    16 {v_2}_1 + 16 {v_2}_2 \\
                    16 {v_2}_1 + 16 {v_2}_2 \\
                \end{bmatrix}
            \end{IEEEeqnarray*}
            With one solution being \[v_2 = \begin{bmatrix} \sqrt 2^{-1} \\ -\sqrt 2^{-1} \end{bmatrix}\]
            For the last eigenvalue \(0\) we get \(v_3 = \begin{bmatrix} 0 \\ 0 \end{bmatrix}\), which formally isn't an eigenvector, since it would be an eigenvector of every matrix, but it is the only solution to the corresponding eigenvalue. \\
            With that we get: 
            \[V^T = \begin{bmatrix}
                \sqrt 2^{-1} & \sqrt 2^{-1} \\
                \sqrt 2^{-1} & -\sqrt 2^{-1} \\
                0 & 0
            \end{bmatrix}\]
            And now we can calculate the product \(U\Sigma V^T\) to verify that it is actually right, which I did, but to be honest, writing all that down with each addition and so on would take a lot of time.
            \qed
        \item[2.]
            We firstly have to center the points, since to calculate the eigenvectors of the covariance Matrix, we can't assume that the mean x and y components are 0. We get:
            \[\overline{x} = 4/3 \wedge \overline{y} = 2/3\]
            With this we can center \(X\):
            \[X_C = \begin{bmatrix}
                        4 - 4/3 & 2 - 2/3 \\
                        \sqrt 2 - 4/3 & 2 \sqrt 2 - 2/3\\
                        -\sqrt 2 - 4/3 & -2 \sqrt 2 - 2/3
                    \end{bmatrix}\]
            And get the covariance Matrix:
            \begin{IEEEeqnarray*}{r'c'l}
                C & = & 1/3 \cdot X_C^TX_C \\
                  & = & 1/3 \cdot \begin{bmatrix}
                                      44/3 & 40/3 \\
                                      40/3 & 56/3
                                  \end{bmatrix}
            \end{IEEEeqnarray*}
            The actual expanded form of this with appropriate steps in between would be way too long to fit on one page. \\
            We actually don't care for the scalar \(1/9\), since the eigenvalues are independent of it since the characteristic polynomial would just be scaled by a scalar, so the zeroes of it are the same. \\
            With this, to get the eigenpairs of the covariance matrix \(C\), we can just get the eigenpairs of \(9 \cdot C\), since it does not matter, since we only need the eigenvectors and they will be normalized later anyways. \\
            We have to solve:
            \begin{IEEEeqnarray*}{r'c'l}
                0 & = & \det(9 \cdot C - \lambda I_{2 \times 2}) \\
                  & = & \det\left(
                        \begin{bmatrix}
                            44 - \lambda & 40 \\
                            40 & 56 - \lambda
                        \end{bmatrix}
                \right) \\
                  & = & (44-x)(56-x) - 40 \cdot 40 \\
                  & = & x^2 - 100x + 864
            \end{IEEEeqnarray*}
            This polynomial has solutions:
            \[\lambda_1 = 50 + 2\sqrt{409} \wedge \lambda_2 = 50 - 2\sqrt{409}\]
            With \(\lambda_1\) being the bigger eigenvalue, we know that the direction of biggest covariance is the eigenvector corresponding to \(\lambda_1\). \\
            On a side note, the eigenvector corresponding to \(\lambda_2\) must be orthogonal to the one corresponding to \(\lambda_1\), since \(C\) is symmetric, this also makes more sense regarding what were calculating. \\
            To get \(\lambda_1\)'s eigenvector, we have to solve:
            \begin{IEEEeqnarray*}{r'c'l}
                \begin{bmatrix}
                    0 \\
                    0
                \end{bmatrix}
                    & = &
                \begin{bmatrix}
                    44 - \lambda_1 & 40 \\
                    40 & 56 - \lambda_1
                \end{bmatrix}
                    \times
                \begin{bmatrix}
                    v_x \\
                    v_y
                \end{bmatrix} \\
                    & = &
                \begin{bmatrix}
                    -(6 + 2 \sqrt{409})v_x + 40v_y \\
                    40v_x + (6 - 2 \sqrt{409})v_y
                \end{bmatrix}
            \end{IEEEeqnarray*}
            Which has infinite solutions, one being:
            \[v = \begin{bmatrix}
                      1 \\
                      1/20(\sqrt{409} - 3)
                  \end{bmatrix}\]
            Which is our eigenvector. And with that being the direction of maximal covariance in \(X\). \\ \\
            On a side note, you see all calculation in this subtask thus far would have been trivial, if \(X\) would have been centered already to begin with, since then we could just have took the first eigenvector in \(V\) from 1.1,this whole ordeal makes the hint \textquote{you may use your results from one subtask for the other} extremely misleading. \\ \\
            Now we have\footnote{I think we don't have to, since for \(s\in \R\) we have \((s \cdot u) \times (s \cdot u^T) = s^2 \cdot u \times u^T\), which means that the projected point would be still on the line, but just with different distance to (0,0), but the lecture suggests that we should.} to normalize \(v\). The length of \(v\) is \(s := \sqrt{1 + 1/400 \cdot (\sqrt{409} - 3)^2} \approx 1.31971352\), which has no further simplification. \\
            Finally we get:
            \[v_n = \frac{1}{s} \cdot \begin{bmatrix}
                      1 \\
                      1/20(\sqrt{409} - 3)
                    \end{bmatrix}\]
            As the normalized version. \\
            According to the lecture, the transformation matrix that projects any point onto the corresponding line is \(v_n \times v_n^T = s^{-2} \cdot v \times v^T\) which is:
            \[\begin{bmatrix}
                \frac{1}{2} + \frac{3}{2 \sqrt{409}} & \frac{10}{\sqrt{409}} \\
                \frac{10}{\sqrt{409}} & \frac{1}{2} - \frac{3}{2 \sqrt{409}}
            \end{bmatrix}\]
            Which is approximately:
            \[\begin{bmatrix}
                0.57417 & 0.49447 \\
                0.49447 & 0.42583
            \end{bmatrix}\]
            \qed
    \end{itemize}
    \pagebreak
    \section{TF-IDF}
    \begin{itemize}
        \item[1.] Vocabulary: ["fast", "car", "highway", "road", "bike", "wheel"]\\
        \item[2.] term frequencies:\\
        \begin{tabular}{c | c c c c}
            \(t\) & \(d = 1\) & \(d = 2\) & \(d = 3\) & \( d = 4\)\\
            \hline
            fast & \(0.2\) & \(0.4\) & \(0.2\) & \(0\) \\
            car & \( 0.4\) & \(0.4\) & \(0\) & \( 0.25\) \\
            highway & \(0.2\) & \(0\) & \(0.2\) & \(0\) \\
            road & \(0.2\) & \(0\) & \(0.4\) & \(0\) \\
            bike & \(0\) & \(0.2\) & \(0\) & \(0.25\) \\
            wheel & \(0\) & \(0\) & \(0.2\) & \(0.5\)
        \end{tabular}\\\\
        \item[3.] document frequency:\\
        \begin{tabular}{c | c | c}
            \(t\) & \(\#d\) containing \(t\) & IDF \\
            \hline
            fast & \(3\) & \(\log{\frac{4}{3}}\) \\
            car & \(3\) & \(\log{\frac{4}{3}}\) \\
            highway & \(2\) & \(\log{\frac{4}{2}} = \log{2}\) \\
            road & \(2\) & \(\log(2)\) \\
            bike & \(2\) & \(\log(2)\) \\
            wheel & \(2\) & \(\log(2)\)
        \end{tabular}\\\\
        \item[4.] calculating TF-IDF vectors:\\
        \begin{tabular}{c | c c c c}
            \(t\) & \(d = 1\) & \(d = 2\) & \(d = 3\) & \( d = 4\)\\
            \hline
            fast & \(0.025\) & \(0.05\) & \(0.025\) & \(0\) \\
            car & \(0.05\) & \(0.05\) & \(0\) & \(0.031\) \\
            highway & \(0.06\) & \(0\) & \(0.06\) & \(0\) \\
            road & \(0.06\) & \(0\) & \(0.120\) & \(0\) \\
            bike & \(0\) & \(0.06\) & \(0\) & \(0.075\) \\
            wheel & \(0\) & \(0\) & \(0.06\) & \(0.151\)
        \end{tabular}\\\\
        Which gives us the following vectors for each \(d\):\\
        \begin{flalign*}
            r(d=1) & = (0.025, 0.05, 0.06, 0.06, 0, 0) & \text{ }\\
            r(d=2) & = (0.05, 0.05, 0, 0, 0.06, 0) & \text{ }\\
            r(d=3) & = (0.025, 0, 0.06, 0.12, 0, 0.06) & \text{ }\\
            r(d=4) & = (0, 0.031, 0 0, 0.075, 0.151) & \text{ }\\
        \end{flalign*}
    \end{itemize}
\end{document}
